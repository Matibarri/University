\documentclass[11pt]{article}
\usepackage[T1]{fontenc}
\usepackage[utf8]{inputenc}
\usepackage[spanish]{babel}
\usepackage{graphicx}
\parindent = 0cm % sangria distancia
%%
\usepackage{amsmath}
\usepackage{amssymb,amsfonts,latexsym,cancel}
\providecommand{\abs}[1]{\lvert#1\rvert}
\providecommand{\norm}[1]{\lVert#1\rVert}
\usepackage [ normalem ]{ ulem }
\useunder { \uline }{ \ul }{}

%%Algoritmos
\usepackage{algpseudocode}
\usepackage{algorithm}
\usepackage{verbatim}
%%
\usepackage[lmargin=2cm,rmargin=2cm,top=2.5cm,bottom=2cm]{geometry}
\usepackage{fancyhdr}
\pagestyle{fancy}
\fancyhead{}
\fancyhead[L]{DEPTO. INGENIERÍA INDUSTRIAL \\ MATIAS BARRIENTOS ZAGAL\\OPTIMIZACION \ II\\2020-2}
\fancyhead[R]{}
%\includegraphics[scale=0.13]{escudoudec}
\fancyfoot{}
\fancyfoot[R]{}
\fancyfoot[L]{}
\renewcommand{\headrulewidth}{0.9pt}
\renewcommand{\footrulewidth}{0.5pt}
%%%%%%%%%%%%%%%%%%%%%%%%%%%%%%%%%%%%%%%%%%%%%%%%%%%%%%%%%%%%%%%%%%%%%%%%%%%%%%%%%%%%%%%
\begin{document}
\begin{center}
\textbf{Tarea 2}
\end{center}

\section{Metaheurística Propuesta}
Se propone un Iterated Local Search (ILS) que se presenta en el \textbf{Algorithm 1} que contiene 4 componentes:
la solución inicial (línea 1); una estrategia de perturbación (línea 5); dos estrategias de búsqueda
local (línea 2 y en las líneas 6–7) y finalmente un criterio de aceptación (línea 8-9). El algoritmo
contempla una solución actual y la mejor solución encontrada, $s$ y $s^*$, respectivamente. En la siguientes
subsecciones se presentan con más detalles la representación, función objetivo, solución inicial, estrategia
de perturbación, estrategias de búsqueda local, criterio de aceptación y parámetros.
\begin{algorithm}[H]
    \caption{Iterated Local Search}
    \label{ils}
    \begin{algorithmic}[1]
    \State $s\gets $ heurística-vecino-mas-cercano-P$()$
    \State $s\gets $3-opt$(s)$
    \State $s* \gets s$
    \Repeat
        \State $s\gets $intercambio-aleatorio$(s)$
        \State $s\gets $2-opt$(s)$
        \State $s\gets $3-opt$(s)$
        \State $s^* \gets $ si $f(s) < f(s^*)$
        \State $s \gets$ criterio-aceptación$(f(s) \leq f(s^*): s,s^*)$
        
    \Until $100$ iteraciones
    \end{algorithmic}
\end{algorithm}
\subsection{Representación}
La represetanción usada es la de permutación, donde cada índice corresponde a una ciudad única,
$s = \{1, . . . , n\}$.

\subsection{Función Objetivo}
Se utiliza la suma de las distancias euclidianas en cada ciudad como se presenta a continuación:
\begin{equation}
    f(s)= \sum_{i=1}^{n-1} distance(s_i , s_{i+1}) + distance(s_n , s_1)
\end{equation}

\subsection{Solución inicial}
Se utilizó la heurística del vecino más cercano, donde el nodo inicial era el depósito, la cual representaba al valor $0$ en las representaciones del problema. Así, la idea del algoritmo es escoger la instalación $i$ más cercana al depósito, luego $i$ pasa a ser la nueva instalación escogida. Repetir el procedimiento de las instalaciones más cercanas hasta que no hayan más instalaciones disponibles.\\
Por otra parte, se añadieron dos iteraciones \textbf{for}, para encontrar un subconjunto de instalaciones que cumplieran con satisfacer el beneficio $P$ de la respectiva instancia. La lógica era aplicar un filtro de los clientes que se estaban cubriendo, buscando no duplicar clientes que son cubiertos por dos instalaciones o más. Luego, se recorre esa lista con los clientes que realmente se cubren y se extraen las instalaciones pertinentes. El procedimiento anterior se presenta en el \textbf{Algorithm 2}.

\begin{algorithm}[H]
    \caption{VecinoMasCercano-P}
    \label{NN}
    \hspace*{} \textbf{Input:} $V$: ciudades, $d_{ij}$ : distancia de la arista $(i,j)$, $P:$ Beneficio \\
    \hspace*{} \textbf{Output:} $s*=\{1, . . . , n\}$
    \begin{algorithmic}[1]
    \State Sol-mejor $  \gets \emptyset$
    \State new-data $  \gets \emptyset$
    \State seen $  \gets \emptyset$
    \State $i \gets 0$ (depósito)
    \State $S \gets S\cup \{i\}$
    \State $S \gets V\setminus\{i\}$
    \State $p \gets 0$ 
    \While{$p<P$}
        \State $k \gets $ mín$\{c_{ij}: j \in V\}$
        \State $S \gets S\cup \{k\}$
        \State $V \gets  V\setminus\{k\}$
        \State $i \gets k$
        \For{$i$ in I-clientes} 
            \State first $ \gets i[0]$ 
            \State second $ \gets i[1]$
            \If{second in $S$}
                \If{first in seen}
                    \State \textbf{continue}
                \State new-data $  \gets i$
                \State seen $  \gets $ first
                \State p $ \gets \#$\{seen\}
                \EndIf
            \EndIf
        \EndFor
        \For{$i$ in new-data}
            \State Sol-mejor $  \gets i[1]$
            \State s $ \gets $ Sol-mejor
        \EndFor
        \If{p $\geq $ P}
            \State \textbf{break}
            
    \EndWhile
    \end{algorithmic}
\end{algorithm}

\subsection{Estrategias de Perturbación}
Se determinan dos enfoques para la estrategia de perturbación. El primero consiste en escoger una
 $i$ aleatoriamente e intercambiar la instalación de esa posición con la instalación de la posición siguiente
$(i + 1)$ de la ruta como se presenta en el \textbf{Algorithm 3}.

\begin{algorithm}[H]
    \caption{perturbacion}
    \label{NN}
    \hspace*{} \textbf{Input:} $s=\{1, . . . , n\}$ \\
    \hspace*{} \textbf{Output:} $s^*$
    \begin{algorithmic}[1]
    \State escoger aleatoriamente una posición $i$
    \State $s^*  \gets $ nueva ruta considerando el intercambio de las instalaciones en las posiciones $i$ y $i+1$
   \end{algorithmic}
\end{algorithm}    

\newpage
El segundo enfoque utilizado es una heurística de mejora que consiste en intercambiar instalaciones
aleatoriamente 30 veces. Así, se tiene el  \textbf{Algorithm 4}

\begin{algorithm}[H]
    \caption{intercambio-aleatorio}
    \label{NN}
    \hspace*{} \textbf{Input:} $s=\{1, . . . , n\}$ \\
    \hspace*{} \textbf{Output:} $s^*$
    \begin{algorithmic}[1]
    \Repeat
        \State escoger aleatoriamente una posición $i$ de $s$
        \State escoger aleatoriamente una posición $j$ de $s$
        \State $s  \gets $ intercambiar las instalaciones de las posiciones $i$ y $j$
    \Until 30 veces
   \end{algorithmic}
\end{algorithm}    
\subsection{Estrategia de Búsqueda Local}
En esta estrategia se utilizan dos enfoques. La primera es la búsqueda local 2-opt, la cual consiste
en recorrer iterativamenete pares de arista e intercambiarlos. En caso de que con el intercambio la ruta
mejore su costo, entonces, se mantiene, caso contrario se deshace. En particular, en esta implementación
se utiliza la versión que si se encuentra una mejora el algoritmo se detiene. El algoritmo antes descrito se
presentan en el  \textbf{Algorithm 5}
\begin{algorithm}[H]
    \caption{\ \ 2-Opt}
    \label{2-opt}
    \hspace*{} \textbf{Input:} $S=\{v_1,v_2,...,v_n\}$ , $c_{ij}$ : pesos de las aristas \\
    \hspace*{} \textbf{Output:} $S'=$ Nueva Ruta 
    \begin{algorithmic}[1]
    \For{$i \gets 1$ hasta $n-2$}
        \For{$j \gets i+1$ hasta $n-2$}
            \If{$c_{v_iv_{i+1}} + c_{v_jv_{j+1}} > c_{v_iv_{i+1}} + c_{v_{i+1}v_{j+1}}$} \Then
                \State Se desconecta $(v_i,v_{i+1})$ y $(v_j,v_{j+1})$
                \State Se reemplaza por $(v_i,v_j)$ y $(v_{i+1},v_{j+1})$
                \State Se actualiza el costo de la solución actual.
            \EndIf
        \EndFor
    \EndFor
    \end{algorithmic}
\end{algorithm}
El segundo enfoque fue una extensión del $2$-Opt, ya que se utilizó el $3$-Opt. Por simplicidad, se evita escribir el algoritmo. Sin embargo, hay que especificar que dicho algoritmo tiene 3 iteraciones de \textbf{for}.
\subsection{Criterio de aceptación}
Si la solución candidata es mejor que la actual, entonces candidata se acepta como solución actual, caso contrario se usa la mejor encontrada, vale decir, $f(s) \leq f(s^*)$.
\subsection{Parámetros}
El número de iteraciones se define en $100$ y los demás datos son respectivos de la instancia. Ya sea $|I|$,$|V|$,$|C|$ y el beneficio $P$. También, las coordenadas entre instalaciones vienen dadas en dos dimensiones, es decir, como un vector $(x_i, y_i)$ 
\section{Resultados Computacionales}
El algoritmo propuesto es implementado en Python 3.7. Por otra parte, las pruebas se realizaron en un Intel(R) Core(TM) i5-7300HQ CPU 2.50 GHz y 16 GB de RAM
(ejecutado en un solo un hilo), en el sistema operativo Windows 10.

\begin{table}[H]
\centering
\caption{Resultados instancias}
\label{tab:my-table}
\begin{tabular}{|c|c|c|c|c|c|c|}
\hline
\textbf{instancias} & \textbf{V} & \textbf{I} & \textbf{costo optimo} & \textbf{ERP} & \textbf{promedio} & \textbf{tiempo} \\ \hline
S1                  & 51         & 15         & 71,33                 & 37,39\%      & 98                & 0,0209          \\ \hline
S2                  & 51         & 15         & 105,98                & 30,21\%      & 138               & 0,1625          \\ \hline
S3                  & 51         & 15         & 172,36                & 2,11\%       & 176               & 0,4478          \\ \hline
S4                  & 51         & 20         & 75,29                 & 2,27\%       & 77                & 0,0209          \\ \hline
S5                  & 51         & 20         & 102,59                & 22,82\%      & 126               & 0,1806          \\ \hline
S6                  & 51         & 25         & 38,49                 & 0,00\%       & 38,49             & 0,0099          \\ \hline
S7                  & 51         & 25         & 82,38                 & 28,67\%      & 106               & 0,1874          \\ \hline
S8                  & 51         & 25         & 140,62                & 8,09\%       & 152               & 0,6781          \\ \hline
S9                  & 52         & 16         & 1378,45               & 62,94\%      & 2246              & 0,0458          \\ \hline
S10                 & 52         & 16         & 2198,91               & 53,53\%      & 3376              & 0,0987          \\ \hline
S11                 & 52         & 21         & 669,76                & 53,79\%      & 1030              & 0,0239          \\ \hline
S12                 & 52         & 21         & 1554,5                & 45,51\%      & 2262              & 0,1002          \\ \hline
S13                 & 52         & 21         & 3910,04               & 7,36\%       & 4198              & 0,6252          \\ \hline
S14                 & 52         & 26         & 572,13                & 62,90\%      & 932               & 0,0505          \\ \hline
S15                 & 52         & 26         & 1240,67               & 45,57\%      & 1806              & 0,2838          \\ \hline
S16                 & 52         & 26         & 2958,78               & 10,89\%      & 3281              & 0,8803          \\ \hline
S17                 & 52         & 16         & 1340,29               & 10,95\%      & 1487              & 0,0449          \\ \hline
S18                 & 52         & 16         & 2291,58               & 71,72\%      & 3935              & 0,2749          \\ \hline
S19                 & 52         & 21         & 878,3                 & 81,03\%      & 1590              & 0,0967          \\ \hline
S20                 & 52         & 21         & 1521,64               & 4,49\%       & 1590              & 0,0970          \\ \hline
S21                 & 52         & 21         & 3590,08               & 4,09\%       & 3737              & 0,4309          \\ \hline
S22                 & 52         & 26         & 479,76                & 93,22\%      & 927               & 0,0219          \\ \hline
S23                 & 52         & 26         & 1334,81               & 29,83\%      & 1733              & 0,1915          \\ \hline
S24                 & 52         & 26         & 2894,96               & 13,20\%      & 3277              & 0,6229          \\ \hline
S25                 & 76         & 23         & 98,94                 & 20,27\%      & 119               & 0,2828          \\ \hline
S26                 & 76         & 23         & 147,89                & 19,01\%      & 176               & 0,8764          \\ \hline
S27                 & 76         & 30         & 86,16                 & 20,71\%      & 104               & 0,1635          \\ \hline
S28                 & 76         & 30         & 119,54                & 51,41\%      & 181               & 2,5626          \\ \hline
S29                 & 76         & 38         & 57,79                 & 34,97\%      & 78                & 0,1800          \\ \hline
S30                 & 76         & 38         & 106,39                & 51,33\%      & 161               & 0,4399          \\ \hline
S31                 & 76         & 38         & 173,59                & 21,55\%      & 211               & 2,1011          \\ \hline
S32                 & 76         & 38         & 20279,16              & 17,18\%  & 23763             & 0,2842          \\ \hline
promedio            & 57,75      & 23,875     & 1583,53625            & 31,84\%      & 1972,234063       & 0,3902          \\ \hline
\end{tabular}
\end{table}

\end{document}
